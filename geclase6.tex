\documentclass{article}
\usepackage{fontspec}
\usepackage{graphicx}
\usepackage{fancyhdr}
\usepackage{geometry}
\usepackage{lipsum}
\usepackage[spanish]{babel}
\usepackage{xcolor}
\usepackage{eso-pic}
\usepackage{tikz}
\usepackage{atbegshi}
\usepackage[most]{tcolorbox} 
\usepackage{listings}
\usepackage{fontawesome5}

\usetikzlibrary{backgrounds}

\setmainfont[
    Path = ./fonts/Poppins/,
    Extension = .ttf,
    UprightFont = *-Regular.ttf,
    BoldFont = *-Bold.ttf,
    ItalicFont = *-Italic.ttf,
    BoldItalicFont = *-BoldItalic.ttf,
    % LetterSpace=5, 
]{Poppins}



\newlength{\borderwidth}
\setlength{\borderwidth}{20pt} 


\geometry{a4paper,
          left=\dimexpr1cm+\borderwidth,
          right=\dimexpr1cm+\borderwidth,
          top=\dimexpr2cm+\borderwidth,
          bottom=\dimexpr0.5cm+\borderwidth,
          headheight=1.5cm,
          headsep=0.5cm}


\pagestyle{fancy}
\fancyhf{}
\fancyhead[L]{\raisebox{0\height}{\includegraphics[height=1cm,keepaspectratio]{logo-header.png}}}
\fancyhead[R]{\raisebox{1.25\height}{\large\textbf{Academia Juvenil de Programación Competitiva}}}
\renewcommand{\headrulewidth}{0pt}


\definecolor{primary}{HTML}{116BB1}
\definecolor{secondary}{HTML}{711E8C}
\definecolor{tertiary}{HTML}{1ECA6C}


\newcommand{\borderOverlay}{
    \begin{tikzpicture}[remember picture, overlay]
        \draw[line width=\borderwidth, color=primary]
            (current page.north west) -- 
            (current page.north east) --
            (current page.south east) --
            (current page.south west) -- cycle;
    \end{tikzpicture}
}

\AddToShipoutPictureBG{\borderOverlay}

\newtcolorbox{container}[1]{ 
    colback=white, 
    colframe=primary,
    coltitle=white,
    title={#1},
    fonttitle=\bfseries, 
    arc=2pt,
    boxrule=2pt, 
    enhanced, 
    breakable, 
    attach boxed title to top left={xshift=5pt, yshift=-5pt}, 
    boxed title style={colback=primary, size=small}, 
}


\lstdefinestyle{cppstyle}{
    language=C++,
    basicstyle=\ttfamily\small,
    keywordstyle=\color{primary},
    commentstyle=\color{tertiary},
    stringstyle=\color{secondary},
    backgroundcolor=\color{white!100},
    breaklines=true,
    breakatwhitespace=true,
    tabsize=4,
    showstringspaces=false,
    captionpos=b
}

\newcommand{\cppfile}[2][]{
    \begin{container}{\faCode \space \space  #1}
        \lstinputlisting[style=cppstyle]{#2}
    \end{container}
}

\newcommand{\inlinecpp}[1]{
    \lstinline[
        language=C++,
        basicstyle=\ttfamily\small\color{primary!80!black},
        keywordstyle=\color{primary},
        stringstyle=\color{secondary},
        commentstyle=\color{tertiary}
    ]|#1|
}


\newcommand{\documentTitle}{Clase 6: STL 1}
\newcommand{\documentSubtitle}{Contenidos}
\newcommand{\documentAuthor}{Francisco Muñoz}
\newcommand{\documentDate}{30 de Agosto}

\begin{document}

\thispagestyle{empty}
\AddToShipoutPictureBG*{}
\begin{center}
    \vspace*{2cm}
    \includegraphics[width=0.75\textwidth]{logo.png} \\[1.5cm]
    {\Huge \textbf{\documentTitle}} \\[0.5cm]
    {\Large \documentSubtitle} \\[1.5cm]
    {\large \documentAuthor} \\[0.5cm]
    {\large \space \space \documentDate}
    \vfill
    {\large Academia Juvenil de Programación Competitiva}
\end{center}
\newpage

\section{Introducción}

En esta clase aprenderemos un concepto fundamental de la \textbf{Standard Template Library (STL)} de C++: las \textbf{estructuras de datos}. 
En particular, veremos dos muy utilizadas en programación competitiva y en la vida real: las \textbf{pilas (stack)} y las \textbf{colas (queue)}.

\section{Índice}

\begin{itemize}
    \item Contenidos
    \begin{itemize}
        \item ¿Qué es la STL?
        \item Pilas (Stack) – LIFO
        \item Colas (Queue) – FIFO
        \item Comparación entre Stack y Queue
    \end{itemize}
    \item Ejercicios Prácticos
\end{itemize}

\section{Contenidos}

\subsection{¿Qué es la STL?}

La \textbf{STL (Standard Template Library)} es una colección de clases y funciones listas para usar en C++.  
Nos permite ahorrar tiempo y escribir código más limpio, en lugar de implementar estructuras de datos desde cero.

Dentro de la STL encontramos:
\begin{itemize}
    \item \texttt{vector}
    \item \texttt{stack}
    \item \texttt{queue}
    \item \texttt{priority\_queue}
    \item \texttt{set}, \texttt{map}, entre otros.
\end{itemize}

En esta clase nos concentraremos en \texttt{stack} y \texttt{queue}.

\vspace{0.5cm}

\subsection{Stack (Pila)}

Una \textbf{pila} es una estructura de datos que funciona bajo la regla \textbf{LIFO} (\textit{Last In, First Out}): el último elemento en entrar es el primero en salir.  
Un ejemplo en la vida real es una pila de platos: siempre tomamos el último que colocamos encima.

\textbf{Operaciones principales:}
\begin{itemize}
    \item \texttt{push(x)}: Inserta un elemento en la pila.
    \item \texttt{pop()}: Elimina el elemento en la cima de la pila.
    \item \texttt{top()}: Devuelve el valor del elemento en la cima.
    \item \texttt{empty()}: Verifica si la pila está vacía.
\end{itemize}

\cppfile[Ejemplo de Stack]{codes/stack.cpp}

\vspace{0.5cm}

\subsection{Queue (Cola)}

Una \textbf{cola} es una estructura de datos que funciona bajo la regla \textbf{FIFO} (\textit{First In, First Out}): el primer elemento en entrar es el primero en salir.  
Un ejemplo en la vida real es la fila del supermercado: el que llega primero es atendido primero.

\textbf{Operaciones principales:}
\begin{itemize}
    \item \texttt{push(x)}: Inserta un elemento al final de la cola.
    \item \texttt{pop()}: Elimina el primer elemento de la cola.
    \item \texttt{front()}: Devuelve el valor del primer elemento.
    \item \texttt{back()}: Devuelve el valor del último elemento.
    \item \texttt{empty()}: Verifica si la cola está vacía.
\end{itemize}

\cppfile[Ejemplo de Queue]{codes/queue.cpp}

\vspace{0.5cm}

\subsection{Stack vs Queue}

\begin{center}
\begin{tabular}{|c|c|c|}
\hline
 & \textbf{Stack (Pila)} & \textbf{Queue (Cola)} \\
\hline
\textbf{Regla} & LIFO (último en entrar, primero en salir) & FIFO (primero en entrar, primero en salir) \\
\hline
\textbf{Ejemplo real} & Pila de platos & Fila en el banco \\
\hline
\textbf{Inserción} & \texttt{push()} agrega arriba & \texttt{push()} agrega al final \\
\hline
\textbf{Eliminación} & \texttt{pop()} quita arriba & \texttt{pop()} quita el primero \\
\hline
\end{tabular}
\end{center}

\vspace{0.5cm}

\section{Ejercicios Prácticos}

\begin{container}{Problema 1 – Último número en una pila}
Escribe un programa que reciba $N$ números y los guarde en una \texttt{stack}. Luego imprime el último número que se ingresó.
\end{container}

\textbf{Input}

Un entero $N$ seguido de $N$ números.

\vspace{0.5em}
\textbf{Output}

El último número ingresado.

\vspace{0.5em}

\begin{container}{Ejemplo - Input}
4 \\
5 7 9 11
\end{container}

\begin{container}{Ejemplo - Output}
11
\end{container}

\vspace{3.5em}


\begin{container}{Problema 2 – El primero en la cola}
Escribe un programa que reciba $N$ palabras y las guarde en una \texttt{queue}. Luego imprime la primera palabra que se ingresó.
\end{container}

\textbf{Input}

Un entero $N$ seguido de $N$ palabras.

\vspace{0.5em}
\textbf{Output}

La primera palabra ingresada.

\vspace{0.5em}

\begin{container}{Ejemplo - Input}
3 \\
perro gato ratón
\end{container}

\begin{container}{Ejemplo - Output}
perro
\end{container}

\vspace{3.5em}


\begin{container}{Problema 3 – Imprimir pila completa}
Escribe un programa que reciba $N$ números y los imprima en orden inverso al que fueron ingresados usando una \texttt{stack}.
\end{container}

\textbf{Input}

Un entero $N$ seguido de $N$ números.

\vspace{0.5em}
\textbf{Output}

Los números en orden inverso.

\vspace{0.5em}

\begin{container}{Ejemplo - Input}
3 \\
10 20 30
\end{container}

\begin{container}{Ejemplo - Output}
30\\
20\\
10
\end{container}

\vspace{3.5em}


\begin{container}{Problema 4 – Imprimir cola completa}
Escribe un programa que reciba $N$ palabras y las imprima en el mismo orden en que fueron ingresadas usando una \texttt{queue}.
\end{container}

\textbf{Input}

Un entero $N$ seguido de $N$ palabras.

\vspace{0.5em}
\textbf{Output}

Las palabras en el mismo orden.

\vspace{0.5em}

\begin{container}{Ejemplo - Input}
3\\
Ana Luis Pedro
\end{container}

\begin{container}{Ejemplo - Output}
Ana\\
Luis\\
Pedro
\end{container}

\vspace{3.5em}


\begin{container}{Problema 5 – Quitar de la pila}
Escribe un programa que reciba $N$ números, los guarde en una \texttt{stack}, luego elimine el último y finalmente imprima el nuevo último número.
\end{container}

\textbf{Input}

Un entero $N$ seguido de $N$ números.

\vspace{0.5em}
\textbf{Output}

El valor que queda en la cima después de quitar el último.

\vspace{0.5em}

\begin{container}{Ejemplo - Input}
4\\ 
1 2 3 4
\end{container}

\begin{container}{Ejemplo - Output}
3
\end{container}



\end{document}