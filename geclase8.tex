\documentclass{article}
\usepackage{fontspec}
\usepackage{graphicx}
\usepackage{fancyhdr}
\usepackage{geometry}
\usepackage{lipsum}
\usepackage[spanish]{babel}
\usepackage{xcolor}
\usepackage{eso-pic}
\usepackage{tikz}
\usepackage{atbegshi}
\usepackage[most]{tcolorbox} 
\usepackage{listings}
\usepackage{fontawesome5}

\usetikzlibrary{backgrounds}

\setmainfont[
    Path = ./fonts/Poppins/,
    Extension = .ttf,
    UprightFont = *-Regular.ttf,
    BoldFont = *-Bold.ttf,
    ItalicFont = *-Italic.ttf,
    BoldItalicFont = *-BoldItalic.ttf,
    % LetterSpace=5, 
]{Poppins}



\newlength{\borderwidth}
\setlength{\borderwidth}{20pt} 


\geometry{a4paper,
          left=\dimexpr1cm+\borderwidth,
          right=\dimexpr1cm+\borderwidth,
          top=\dimexpr2cm+\borderwidth,
          bottom=\dimexpr0.5cm+\borderwidth,
          headheight=1.5cm,
          headsep=0.5cm}


\pagestyle{fancy}
\fancyhf{}
\fancyhead[L]{\raisebox{0\height}{\includegraphics[height=1cm,keepaspectratio]{logo-header.png}}}
\fancyhead[R]{\raisebox{1.25\height}{\large\textbf{Academia Juvenil de Programación Competitiva}}}
\renewcommand{\headrulewidth}{0pt}


\definecolor{primary}{HTML}{116BB1}
\definecolor{secondary}{HTML}{711E8C}
\definecolor{tertiary}{HTML}{1ECA6C}


\newcommand{\borderOverlay}{
    \begin{tikzpicture}[remember picture, overlay]
        \draw[line width=\borderwidth, color=primary]
            (current page.north west) -- 
            (current page.north east) --
            (current page.south east) --
            (current page.south west) -- cycle;
    \end{tikzpicture}
}

\AddToShipoutPictureBG{\borderOverlay}

\newtcolorbox{container}[1]{ 
    colback=white, 
    colframe=primary,
    coltitle=white,
    title={#1},
    fonttitle=\bfseries, 
    arc=2pt,
    boxrule=2pt, 
    enhanced, 
    breakable, 
    attach boxed title to top left={xshift=5pt, yshift=-5pt}, 
    boxed title style={colback=primary, size=small}, 
}


\lstdefinestyle{cppstyle}{
    language=C++,
    basicstyle=\ttfamily\small,
    keywordstyle=\color{primary},
    commentstyle=\color{tertiary},
    stringstyle=\color{secondary},
    backgroundcolor=\color{white!100},
    breaklines=true,
    breakatwhitespace=true,
    tabsize=4,
    showstringspaces=false,
    captionpos=b
}

\newcommand{\cppfile}[2][]{
    \begin{container}{\faCode \space \space  #1}
        \lstinputlisting[style=cppstyle]{#2}
    \end{container}
}

\newcommand{\inlinecpp}[1]{
    \lstinline[
        language=C++,
        basicstyle=\ttfamily\small\color{primary!80!black},
        keywordstyle=\color{primary},
        stringstyle=\color{secondary},
        commentstyle=\color{tertiary}
    ]|#1|
}


\newcommand{\documentTitle}{Clase 8}
\newcommand{\documentSubtitle}{Contenidos de la Clase}
\newcommand{\documentAuthor}{Francisco Muñoz}
\newcommand{\documentDate}{13 de Septiembre, 2025}

\begin{document}

\thispagestyle{empty}
\AddToShipoutPictureBG*{}
\begin{center}
    \vspace*{2cm}
    \includegraphics[width=0.75\textwidth]{logo.png} \\[1.5cm]
    {\Huge \textbf{\documentTitle}} \\[0.5cm]
    {\Large \documentSubtitle} \\[1.5cm]
    {\large \documentAuthor} \\[0.5cm]
    {\large \space \space \documentDate}
    \vfill
    {\large Academia Juvenil de Programación Competitiva}
\end{center}
\newpage

\section{Introducción}

Este documento contiene los contenidos que deberían ser vistos durante la \textbf{Clase N°8}.

\section{Índice}

\begin{itemize}
    \item Contenidos
    \begin{itemize}
        \item La estructura de datos \texttt{set}
        \item Operaciones principales en \texttt{set}
        \item Ejemplos de uso de \texttt{set}
        \item La estructura de datos \texttt{map}
        \item Operaciones principales en \texttt{map}
        \item Ejemplos de uso de \texttt{map}
        \item Comparación entre \texttt{set} y \texttt{map}
        \item Problemas típicos
    \end{itemize}
\end{itemize}

\section{Contenidos}

\subsection{La estructura de datos \texttt{set}}

\textbf{¿Qué es un \texttt{set}?}

Un \texttt{set} es una estructura de datos que permite almacenar elementos de forma:
\begin{itemize}
    \item \textbf{Ordenada:} los elementos se mantienen en orden ascendente automáticamente.
    \item \textbf{Única:} no se permiten elementos repetidos.
    \item \textbf{Eficiente:} inserción, búsqueda y borrado se realizan en tiempo \texttt{O(log n)}.
\end{itemize}

En C++, \texttt{set} está implementado internamente con un \textit{árbol balanceado} (generalmente un red-black tree).

\textbf{Declaración básica:}

\cppfile[Ejemplo]{codes/set1.cpp}

\textbf{Operaciones principales:}
\begin{itemize}
    \item \texttt{s.insert(x)} : Inserta el elemento \texttt{x}.
    \item \texttt{s.erase(x)} : Elimina el elemento \texttt{x}.
    \item \texttt{s.count(x)} : Devuelve 1 si el elemento está, 0 en caso contrario.
    \item \texttt{s.find(x)} : Devuelve un iterador al elemento \texttt{x} si existe, o \texttt{s.end()} si no.
    \item Recorrido con iteradores o con bucles \texttt{for}.
\end{itemize}

\textbf{Ejemplo de inserción y recorrido:}

\cppfile[Ejemplo]{codes/set2.cpp}

\textbf{Ejemplo de eliminación y búsqueda:}

\cppfile[Ejemplo]{codes/set3.cpp}

\vspace{1.5cm}

\subsection{Ejemplos de uso de \texttt{set}}

\begin{itemize}
    \item \textbf{Eliminar duplicados de una lista:} insertar todos los números en un \texttt{set} y luego recorrerlo.
    \cppfile[Ejemplo]{codes/set_ejemplo1.cpp}

    \item \textbf{Verificar existencia rápida:} consultar si un número está en un conjunto.
    \cppfile[Ejemplo]{codes/set_ejemplo2.cpp}
\end{itemize}

\vspace{1.5cm}

\subsection{La estructura de datos \texttt{map}}

\textbf{¿Qué es un \texttt{map}?}

Un \texttt{map} es una estructura de datos que almacena pares \textbf{clave-valor}.  
Cada clave es única, está ordenada y permite acceder rápidamente a su valor asociado.

\textbf{Características principales:}
\begin{itemize}
    \item Claves no repetidas.
    \item Claves ordenadas automáticamente.
    \item Acceso, inserción y borrado en \texttt{O(log n)}.
\end{itemize}

\textbf{Declaración básica:}

\cppfile[Ejemplo]{codes/map1.cpp}

\textbf{Operaciones principales:}
\begin{itemize}
    \item \texttt{m[key] = value} : Inserta o actualiza el valor asociado a \texttt{key}.
    \item \texttt{m.erase(key)} : Elimina el par asociado a \texttt{key}.
    \item \texttt{m.count(key)} : Verifica si existe una clave (1 o 0).
    \item \texttt{m.find(key)} : Devuelve un iterador al par si la clave existe.
\end{itemize}

\textbf{Ejemplo de inserción y acceso:}

\cppfile[Ejemplo]{codes/map2.cpp}

\textbf{Ejemplo de eliminación y búsqueda:}

\cppfile[Ejemplo]{codes/map3.cpp}

\vspace{1.5cm}

\subsection{Ejemplos de uso de \texttt{map}}

\begin{itemize}
    \item \textbf{Contador de frecuencias:} cuántas veces aparece cada número en una lista.
    \cppfile[Ejemplo]{codes/map_ejemplo1.cpp}

    \item \textbf{Agenda:} almacenar pares (nombre → número de teléfono).
    \cppfile[Ejemplo]{codes/map_ejemplo2.cpp}
\end{itemize}

\vspace{1.5cm}

\subsection{Comparación entre \texttt{set} y \texttt{map}}

\begin{itemize}
    \item Un \texttt{set} guarda elementos únicos en orden.
    \item Un \texttt{map} guarda pares clave-valor, con claves únicas en orden.
    \item Ambos ofrecen operaciones en \texttt{O(log n)}.
\end{itemize}



\end{document}