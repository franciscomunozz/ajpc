\documentclass{article}
\usepackage{fontspec}
\usepackage{graphicx}
\usepackage{fancyhdr}
\usepackage{geometry}
\usepackage{lipsum}
\usepackage[spanish]{babel}
\usepackage{xcolor}
\usepackage{eso-pic}
\usepackage{tikz}
\usepackage{atbegshi}
\usepackage[most]{tcolorbox} 
\usepackage{listings}
\usepackage{fontawesome5}

\usetikzlibrary{backgrounds}

\setmainfont[
    Path = ./fonts/Poppins/,
    Extension = .ttf,
    UprightFont = *-Regular.ttf,
    BoldFont = *-Bold.ttf,
    ItalicFont = *-Italic.ttf,
    BoldItalicFont = *-BoldItalic.ttf,
    % LetterSpace=5, 
]{Poppins}



\newlength{\borderwidth}
\setlength{\borderwidth}{20pt} 


\geometry{a4paper,
          left=\dimexpr1cm+\borderwidth,
          right=\dimexpr1cm+\borderwidth,
          top=\dimexpr2cm+\borderwidth,
          bottom=\dimexpr0.5cm+\borderwidth,
          headheight=1.5cm,
          headsep=0.5cm}


\pagestyle{fancy}
\fancyhf{}
\fancyhead[L]{\raisebox{0\height}{\includegraphics[height=1cm,keepaspectratio]{logo-header.png}}}
\fancyhead[R]{\raisebox{1.25\height}{\large\textbf{Academia Juvenil de Programación Competitiva}}}
\renewcommand{\headrulewidth}{0pt}


\definecolor{primary}{HTML}{116BB1}
\definecolor{secondary}{HTML}{711E8C}
\definecolor{tertiary}{HTML}{1ECA6C}


\newcommand{\borderOverlay}{
    \begin{tikzpicture}[remember picture, overlay]
        \draw[line width=\borderwidth, color=primary]
            (current page.north west) -- 
            (current page.north east) --
            (current page.south east) --
            (current page.south west) -- cycle;
    \end{tikzpicture}
}

\AddToShipoutPictureBG{\borderOverlay}

\newtcolorbox{container}[1]{ 
    colback=white, 
    colframe=primary,
    coltitle=white,
    title={#1},
    fonttitle=\bfseries, 
    arc=2pt,
    boxrule=2pt, 
    enhanced, 
    breakable, 
    attach boxed title to top left={xshift=5pt, yshift=-5pt}, 
    boxed title style={colback=primary, size=small}, 
}


\lstdefinestyle{cppstyle}{
    language=C++,
    basicstyle=\ttfamily\small,
    keywordstyle=\color{primary},
    commentstyle=\color{tertiary},
    stringstyle=\color{secondary},
    backgroundcolor=\color{white!100},
    breaklines=true,
    breakatwhitespace=true,
    tabsize=4,
    showstringspaces=false,
    captionpos=b
}

\newcommand{\cppfile}[2][]{
    \begin{container}{\faCode \space \space  #1}
        \lstinputlisting[style=cppstyle]{#2}
    \end{container}
}

\newcommand{\inlinecpp}[1]{
    \lstinline[
        language=C++,
        basicstyle=\ttfamily\small\color{primary!80!black},
        keywordstyle=\color{primary},
        stringstyle=\color{secondary},
        commentstyle=\color{tertiary}
    ]|#1|
}


\newcommand{\documentTitle}{Clase 3}
\newcommand{\documentSubtitle}{Contenidos de la Clase}
\newcommand{\documentAuthor}{Francisco Muñoz}
\newcommand{\documentDate}{9 de Agosto, 2025}

\begin{document}

\thispagestyle{empty}
\AddToShipoutPictureBG*{}
\begin{center}
    \vspace*{2cm}
    \includegraphics[width=0.75\textwidth]{logo.png} \\[1.5cm]
    {\Huge \textbf{\documentTitle}} \\[0.5cm]
    {\Large \documentSubtitle} \\[1.5cm]
    {\large \documentAuthor} \\[0.5cm]
    {\large \space \space \documentDate}
    \vfill
    {\large Academia Juvenil de Programación Competitiva}
\end{center}
\newpage

\section{Introducción}

Este documento contiene los contenidos que deberían ser vistos durante la \textbf{Clase N°3}.

\section{Índice}

\begin{itemize}
    \item Contenidos
    \begin{itemize}
        \item El tipo de dato booleano
        \item Utilidad
        \item Operadores lógicos
        \item Condicionales
    \end{itemize}
\end{itemize}

\section{Contenidos}

\subsection{El tipo de dato booleano}

\textbf{¿Qué es un valor booleano?}

Un valor booleano es un tipo de dato que solo puede tener dos posibles valores: \texttt{true} (verdadero) o \texttt{false} (falso). Este tipo de dato es muy útil para representar condiciones lógicas y controlar el flujo de un programa mediante estructuras como \texttt{if}, \texttt{while}, entre otras.

\textbf{Declaración de variables booleanas:}

Para declarar una variable booleana se utiliza la palabra clave \texttt{bool}. Ejemplos:

\cppfile[Ejemplo]{codes/booleano1.cpp}

\textbf{Valores equivalentes:}

En C++, cualquier número distinto de cero se interpreta como \texttt{true}, y el cero como \texttt{false}. Esto permite usar expresiones numéricas como condiciones booleanas.

\cppfile[Ejemplo]{codes/booleano3.cpp}

\vspace{1.5cm}

\vspace{1.5cm}


\textbf{Uso común:}

Las variables booleanas se utilizan principalmente para evaluar condiciones a través de la sentencia \texttt{if} que será vista mas adelante.


\textbf{Operadores lógicos:}
\begin{itemize}
    \item \texttt{==}  : Igualdad
    \item \texttt{!=}  : Diferente
    \item \texttt{\&\&} : Y lógico (AND)
    \item \texttt{||}  : O lógico (OR)
    \item \texttt{!}   : Negación lógica (NOT)
\end{itemize}

Estos operadores permiten construir expresiones más complejas que devuelven valores booleanos.

\textbf{Operadores lógicos vs. operadores aritméticos:}

En C++ existen distintos tipos de operadores que cumplen funciones muy diferentes:

\begin{itemize}
    \item \textbf{Operadores aritméticos:} se utilizan para realizar operaciones matemáticas como suma, resta, multiplicación, etc.
    \begin{itemize}
        \item \texttt{+}  : Suma
        \item \texttt{-}  : Resta
        \item \texttt{*}  : Multiplicación
        \item \texttt{/}  : División
        \item \texttt{\%} : Módulo (residuo de la división)
    \end{itemize}
    
    \item \textbf{Operadores lógicos y de comparación:} se utilizan para construir condiciones que devuelven valores booleanos (\texttt{true} o \texttt{false}).
    \begin{itemize}
        \item \texttt{==}  : Igualdad (compara si dos valores son iguales)
        \item \texttt{!=}  : Diferente (compara si dos valores son distintos)
        \item \texttt{<}   : Menor que
        \item \texttt{<=}  : Menor o igual que
        \item \texttt{>}   : Mayor que
        \item \texttt{>=}  : Mayor o igual que
        \item \texttt{\&\&} : Y lógico (AND) — Ambas condiciones deben ser verdaderas
        \item \texttt{||}  : O lógico (OR) — Al menos una condición debe ser verdadera
        \item \texttt{!}   : Negación lógica (NOT) — Invierte el valor lógico
    \end{itemize}
\end{itemize}

\textbf{Ejemplo comparando operadores:}

\cppfile[Ejemplo]{codes/operadores_logicos.cpp}



\vspace{1.5cm}

\subsection{Condicionales en C++}

\textbf{¿Qué es una estructura condicional?}

Una estructura condicional permite que un programa tome decisiones durante su ejecución. Esto significa que puede ejecutar ciertas instrucciones solo si se cumple una determinada condición lógica (booleana).

\vspace{0.5em}
\textbf{La estructura \texttt{if}:}

La forma más básica de una estructura condicional es el uso de \texttt{if}. Si la condición dentro del paréntesis es verdadera, se ejecuta el bloque de código encerrado entre llaves.

\cppfile[Ejemplo]{codes/condicional_if.cpp}

\vspace{0.5em}
\textbf{La estructura \texttt{if - else}:}

Cuando queremos realizar una acción si se cumple la condición, y otra acción diferente si no se cumple, utilizamos \texttt{if - else}.

\cppfile[Ejemplo]{codes/condicional_if_else.cpp}

\vspace{0.5em}
\textbf{La estructura \texttt{if - else if - else}:}

Se utiliza cuando hay múltiples condiciones posibles. El programa evalúa las condiciones en orden, y ejecuta el primer bloque cuya condición sea verdadera. Si ninguna se cumple, ejecuta el bloque final \texttt{else}.

\cppfile[Ejemplo]{codes/condicional_if_elseif_else.cpp}

\vspace{0.5em}


\end{document}